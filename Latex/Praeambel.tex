%----- Schrift und Sprache --------------
\usepackage[ngerman]{babel}				%%% Sprachpaket für Deutsche Sprache

\usepackage[utf8]{inputenc}				%%% Verwendung von Umlauten und Kodierung dieser, sodass Latex es kompilieren kann

\usepackage[T1]{fontenc} 				%%% Silbentrennung bei Sonderzeichen

\usepackage{textcomp}					%%% Zusätzlihe Symbolzeichen

\usepackage{courier}					%%% Schriftart-Paket

\usepackage{lmodern}					%%% (=Lain Modern) Ausgabe von flüssigem, wenig verpixeltem Text

\usepackage{romanbar}					%%% Verwendung römischer Zahlen

\usepackage{siunitx}					%%% Setzt Einheiten und Zahlen korrekt - nützlich besonders bei naturwissenschaftlichen Arbeiten

%------ Seite einrichten ----------------

\usepackage[includehead,includefoot, left=3.0cm, right=2.5cm, top=2.5cm, bottom=2.5cm]{geometry}								%%% Gestaltung der Seitenrändern, Kopf- und Fußzeile

\usepackage[onehalfspacing]{setspace} 	%%% Zeilenbstand anpassen, Hier:1,5-fach

\usepackage{scrextend}					%%% Erweitert die Möglichkeiten von bestimmten Dokumentklassen


%----- Mathemaische Formeln ------------
\usepackage{amsmath} 					%%% Erweitert den möglichen Formelsatz
\usepackage{amssymb}					%%% Fügt weiter mathematische Symbole und Pfeile ein

%----- Tabellen ------------------------

\usepackage{booktabs}					%%% Tabellen formatieren - horizotale Linien

\usepackage{multirow}					%%% Tabellen foratieren - mehrzeiliger Text in einer Zelle

\usepackage{colortbl}					%%% Um Zellen in Tabellen einzufärben
\usepackage{xcolor} 					%%% Bindet ein Farb-Paket ein 

%%% Vergrößerung des Platzes in Tabellen, besonders nach \hline notwenig
\newcommand\T{\rule{0pt}{2.6ex}}       	% oben
\newcommand\B{\rule[-1.2ex]{0pt}{0pt}} 	% unten  

\usepackage{subfigure}					%%% Zur beschriftung mehrerer Bilder mit dem gleichen Titel

\usepackage{float}						%%% Biler und Tabellen werden horizontal umflossen

\usepackage[export]{adjustbox}			%%% Ausrichtung von Figuren links, rechts, zentriert, innen und außen

%----- Zitieren und Referenzen -----------------

\usepackage[sort]{natbib}				%%% Zitationstil festlegen (Hier: Autor, Jahr)

\usepackage[german=quotes]{csquotes} 	%%% korrektes Anzeigen von Zitaten

\usepackage{makeidx}
\makeindex
%----- Hyperlinks -----------------------------
\usepackage{hyperref}					%%% Einbinden von Links und Verweisen im Dokument
\hypersetup{colorlinks=true} 			%%% Ermöglicht das einfärben von Links
\hypersetup{urlcolor=blue}				%%% URL werden blau angezeigt
\hypersetup{citecolor=black}			%%% Zitate werden schwarz angezeigt
\hypersetup{linkcolor=black}			%%% Verweise werden schwarz angezeigt

%----- Importieren -----

\usepackage{graphicx}					%%% Einbinden von Graiken und Bildern

\usepackage{wrapfig}					%%% um Grafiken und Bilder textumflossen einzubinden - muss als letztes Paket eingebunden werden, da es zum Teil auf andere Pakete zugreift
\usepackage{pdfpages}					%%% PDF können importiert werden



%%%%%%%%%%%%%%%%%%%%%%%%%%%%%%%%%%%%%%%%%%%%%%%%%%%%%%%%%%%%%%%%%%%%%%%%%%%%%%%%%%%%%%
% Alles, was jetzt folgt definiert Kopf- und Fußzeilen neu und bindet so das HTWK-Design ein.
%
%\clearscrheadfoot					% Löscht alle Kopf- und Fußzeilen
%\setfootwidth{\textwidth}			% Definiert die Breite von Kopf und Fußzeile
%\ohead{\hspace{-6cm} \raisebox{6cm}{\includegraphics[width=10cm]{GELB} }}
%									%  Fügt das gelbe Rechteck im äußeren Bereich der Fußzeile ein
%\ihead{\raisebox{10cm}{\headmark}}	% Fügt im inneren Bereich klein die Kapitelüberschriften ein
%\ofoot{\raisebox{\baselineskip}{\large {\textbf{\pagemark}}}}%[24mm][12mm]
%									% Setzt unten, außen die Seitenzahl

