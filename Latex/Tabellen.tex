\section{Vorlage Tabelle}
\label{sec:Tabellen}
\subsection{Einfache Tabelle - \textit{tabular}}

\begin {tabular} [h] {c|l|p{5cm}}
Was wird betrachtet? & Mathematische Formel & Anwendungsbereich\\\hline
Spalte 1 & Spalte 2 & \dots \\
Spalte 3 & Spalte 4 & \dots \\
\end{tabular}
\captionof{table}{Das ist eine Vorlagen-Beispiel für eine einfache Tabelle}
\label{tab:e_Tabelle}

\subsection{Aufzählung mit Anstrich - \textit{itemize}}

\begin{itemize}
\item Erster Anstrich
\item Zweiter Anstrich
\item[$\star$]\dots
\end{itemize}


\subsection{nummerierte Aufzählung \textit{enumerate}}
\begin{enumerate}
\item Erster Anstrich
\item[$\star$] Zweiter Anstrich
\end{enumerate}

\subsection{Tabstop Umgebung \textit{tabbing}}
\begin{tabbing}
\hspace{4cm}\=\hspace{4cm}\=\kill
 Erster \>  Fakt\> deutsch \\ 
 first \>  fact \> englisch
\end{tabbing} 

\subsection{Ausgerichtete Aufzählung - \textit{align}}

\begin {align}
Formel\ 1\ & mit\ 2\ Unbekannten & = Ergebnis 1\\
Formel\ 2\ & &=Ergebnis 2
\end{align}

