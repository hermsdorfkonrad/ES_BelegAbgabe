\chapter{Umsetzung des Projektes}
\section{Komponenten und Aufbau}
\label{sec:Peri}
Das Projekt wird auf einem Arduino Micro umgesetzt. Dieser ist ein Entwicklerboard, basierend auf dem Mikrocontroller ATmega32U4. Er wird mit 16MHz getaktet und 5V betrieben. Der Programmspeicher umfasst 32KB. Weiterhin besitzt der Controller 2.5KB SRAM als Arbeitsspeicher sowie 1KB EEPROM für dauerhaftes Speichern von Werten.
Die Steuerung des Spiels erfolgt über Joysticks, dessen Position über Potentiometer in eine Spannung abgebildet wird. 
Die Anzeige erfolgt über ein 4-Zeilen Display (Modell 2004) mit parallelem Interface. Ein FC-113 I2C-Brückenchip wandelt die seriell über I2C gesendeten Displaykommandos in das parallel Interface um. Das Programm wird mittels Programmer EvUSBasp auf den Controller geladen. Im Folgenden sind die wichtigsten verwendeten Komponenten aufgezählt:
\begin{itemize}
\setlength\itemsep{-0.5em}
\item Arduino Micro
\item I2C Displayinterface FC-113 
\item 4-Zeilen LC-Display Modell 2004
\item 2 $\times$ Potentiometer-Joystick
\item EvUSBasp USB-Programmer
\end{itemize}

Abbildung \ref{fig:Aufbau} zeigt den Aufbau des Projektes.
\begin{figure}[H]
\includegraphics[width=\textwidth]{./Bilder/Aufbau.jpg}
\caption{Aufbau des Projektes}
\label{fig:Aufbau}
\end{figure}

Die nachfolgende Tabelle listet die Pinbelegung des Arduino Micro in diesem Projekt. Die Massepins und Versorgungsspannungen wurde für bessere Übersichtlichkeit ausgelassen.

\begin{table}[H]
\begin{center}
\caption{Pinbelegung}
\begin{tabular}{|c|c|}
\hline 
Komponente & Pin am Arduino \\ 
\hline
\hline 
Joystick 1 - Teilerspannung & A5 \\ 
\hline 
Joystick 2 - Teilerspannung & A0 \\ 
\hline 
Display - I2C SDA & D2 \\ 
\hline 
Display - I2C SCL & D3 \\ 
\hline 
\end{tabular}
\end{center}
\end{table}
%Pinout nennen

\section{Entwicklungsumgebung}
\label{sec:IDE}
Die zur Programmierung des ATmega32U4 unter Ubuntu Linux notwendigen Programme werden mittels Paketverwaltung \glqq apt\grqq\ installiert:
\begin{itemize}
\setlength\itemsep{-0.5em}
\item Es wird der ATmega-kompatible C-Compiler \glqq avr-gcc\grqq\ genutzt.
\item Für den Flash-Vorgang ist die Programmiersoftware \glqq avrdude\grqq\ notwendig.
\item Die automatische Kompilation und das Linken wird durch die Software \glqq Make\grqq\ verwaltet. Dazu wird ein sog. \glqq Makefile\grqq\ erstellt (siehe Abschnitt \ref{sec:Make}).
\item Zur Erstellung des Quelltextes wird der Editor \glqq Visual Studio Code\grqq\ eingesetzt.
\item Der Programmer USBasp konnte ohne Installation von zusätzlichen Treibern eingesetzt werden.
\end{itemize}
\noindent Die Wahl des Editors fiel auf Visual Studio Code, da dieser Abhängigkeiten, sowie C-Syntax überprüfen kann. Die Integration von Intellisense als Algorithmus hinter der Autovervollständigung ermöglicht effizientes Verfassen von Quelltexten. Die nachfolgende Abbildung zeigt die Oberfläche von Visual Studio Code.

\begin{figure}[H]
\includegraphics[width=\textwidth]{./Bilder/Code.png}
\caption{Oberfläche von Visual Studio Code}
\end{figure}

\section{Programmstruktur}
\subsection{Funktionsumfang}
\label{sec:Funktionsumfang}
Das Programm lässt sich grob in die Aufgabenbereiche Spielschleife, Zeichenroutinen und Verwaltung der Peripherie einteilen. Nach einer anfänglichen Initialisierungsphase, wird in der Spielschleife zyklisch die Position der Joysticks über den ADC-Wert abgefragt. Dieser Wert wird genutzt, um die Position der Ankerpunkte der Schläger im nächsten Bild zu berechnen. Bei Berechnung der Ballposition wird beachtet, ob der Ball hinter einem Schläger gelandet ist oder eine Wand berührt hat in erstem Fall wird eine Siegesnachricht angezeigt, in letzterem Fall die y-Richtung des Balls umgekehrt. Bei Berührung eines Schlägers wird die x-Richtung umgekehrt.\\
Die Größe der Bewegung je Bild für Ball und Schläger wird außerdem proportional zur verstrichenen Zeit zwischen zwei Bildern festgelegt. Dazu wird Timer 1 des ARmega32U4 zur Auslösung einer Interrupt-Service-Routine eingesetzt, welche eine Variable inkrementiert, die die Verstrichenen Mikrosekunden seit Timer-Start speichert.\\
Die Ansteuerung des Displays erfolgt über den Laborunterlagen entnommenen Programmcode. Dieser erlaubt die Programmierung eigener Schriftzeichen im Display. Das wurde genutzt um je Displayzeile 4 verschiedene Schlägerpositionen darstellen zu können.
Eine ausführliche Dokumentation aller Funktionen findet sich im Anhang \ref{cha:Doxy} in Form einer Doxygen-generierten Quelltextbeschreibung. Der Ablauf des Programms ist vereinfacht in Abbildung \ref{fig:PAP} unter Vernachlässigung der Wartezeiten und Übergabewerte der Funktionen dargestellt.

\begin{figure}[H]
\centering
\includegraphics[height=0.8\textheight]{./Bilder/Programmablauf.pdf}
\centering\caption{Vereinfachter Ablauf des Programmes}
\label{fig:PAP}
\end{figure}

\subsection{Modularisierung des Codes}
\label{sec:Modularisierung}
Um die Wartbarkeit des Codes zu erhöhen, wird die Funktionalität in mehrere Teilfunktionen ausgelagert, welche wiederum thematisch in einzelne Quelltextdateien einsortiert werden. 
Tabelle \ref{tab:Code} listet die Quelltextdateien und deren Inhalt. Sofern eine zur Headerdatei zugehörige \texttt{.c}-Datei existiert, bezieht sich die Tabelle auf beide Dateien.

\begin{center}
\begin{table}[H]
\caption{Inhalt der Quelltextdateien}
\label{tab:Code}
\end{table}
\begin{longtable}{|p{0.2\textwidth}|p{0.7\textwidth}|}
\hline 
Quelltextdatei & Inhalt \\ 
\hline 
\hline
\texttt{main.c} & 	\begin{itemize}
					\setlength\itemsep{-0.5em}
					\item Initialisierung des Timers,Peripherie und Spielvariablen
					\item Pong-Dauerschleife
					\end{itemize} 
\\ 
\hline 
\texttt{config.h} &	\begin{itemize}
					\setlength\itemsep{-0.5em}
					\item Präprozessormakros zur Konfiguration des Spiels
					\item Adressen für Pins und Displayspeicher
					\end{itemize} 
\\ 
\hline 
\texttt{pong.h} &	\begin{itemize}
					\setlength\itemsep{-0.5em}
					\item Strukturvariablen für Ball und Schläger
					\item Routinen zum Zeichnen des Spiels und Berechnen des Spielgeschehens
					\end{itemize} 
\\ 
\hline 
\texttt{customFont.h} &	\begin{itemize}
					\setlength\itemsep{-0.5em}
					\item Bitmuster der selbstdefinierten Zeichen für das Display
					\end{itemize} 
\\ 
\hline 
\texttt{misc.h} &	\begin{itemize}
					\setlength\itemsep{-0.5em}
					\item Funktionen zum Hochladen und Ausgeben der selbstdefinierten Zeichen auf das Display
					\item ADC-Initialisierung und -Lesevorgang
					\item Timer-Setup und zugehörige Interrupt-Service-Routine
					\end{itemize} 
\\ 
\hline
\texttt{i2clcd.h} &	\begin{itemize}
					\setlength\itemsep{-0.5em}
					\item bereitgestellter Quelltext
					\item Grundlegende Routinen zum Ansteuern des Displays mit I2C-Expander
					\end{itemize} 
\\ 
\hline 
\texttt{twimaster.h} &	\begin{itemize}
					\setlength\itemsep{-0.5em}
					\item bereitgestellter Quelltext
					\item Routinen zur Kommunikation mit einem I2C-Slave, wie dem Display
					\end{itemize} 
\\ 
\hline 
\end{longtable}
\end{center}

\section{Automatische Maschinencodeerstellung mittels Makefile}
\label{sec:Make}
Das Übersetzen des Projekts wird durch Make gesteuert. Make selbst
wird mithilfe eines sogenannten \textquotedbl Makefile\textquotedbl{}
parametriert. Das Makefile definiert allgemein betrachtet Regeln,
nach denen die Buildreihenfolge festgelegt wird. Genauer handelt es
sich bei den Regeln um Abhängigkeiten. Anhand der in den Regeln beschriebenen
Abhängigkeiten weiß Make, welche Dateien für den Buildvorgang zu welchem
Zeitpunkt verfügbar sein müssen.

Darüber hinaus prüft Make auch die Zeitstempel der Quelldateien und
stellt auf diesem Weg sicher, dass nur die Quelldateien (erneut) übersetzt
werden, welche sich seit dem letzten Build geändert haben. Diese Funktion
macht den Buildvorgang in der Praxis deutlich effizienter.

\subsection*{Funktionen des Makefile}

Im Makefile des Projekts sind die wichtigsten Funktionen für den praktischen
Einsatz implementiert. 
\begin{lyxcode}
make

make~all
\end{lyxcode}
Durch einfaches ausführen von make im Projektverzeichnis wird das
gesamte Projekt durch den avr-gcc übersetzt und verlinkt. Dabei wird
eine {*}.elf Datei der Firmware und ein Speicherabbild für den Mikrocontroller
erstellt. Weiterhin wird das Abbild des EEPROM-Inhalts generiert,
dieser ist in diesem Projekt jedoch leer.
\begin{lyxcode}
make~program
\end{lyxcode}
Durch make program wird das erstellte Speicherabbild direkt mithilfe
des USB Flash Programmers in den Flash Speicher des Mikrocontrollers
geschrieben. Es ist nicht erforderlich zuerst make und anschließend
make program auszuführen.
\begin{lyxcode}
make~size
\end{lyxcode}
Ausgabe der Speicherabbildgrößen um zu bestimmen, dass die Firmware
in den Speicher des Mikrocontrollers passt beziehungsweise wieviel
Speicherplatz noch bleibt für weitere Funktionen.
\begin{lyxcode}
make~clean
\end{lyxcode}
Um das Projektverzeichnis von temporären Dateien und Build-Artefakten
zu bereinigen wird make clean ausgeführt. Dabei werden einerseits
die Verzeichnisse mit den Objektdateien und Binaries gelöscht, andererseits
werden auch die intermediären Dateien vom Übersetzen der in Latex
verfassten Dokumendation entfernt.
\begin{lyxcode}
make~infos
\end{lyxcode}
Für das Projekt an sich erfüllt make infos keinen Zweck. Es dient
lediglich als Target für Test- und Debuggingzwecke des Makefile selbst.
Der Entwickler hat auf diesem Weg die Möglichkeit erstellte Variablenlisten
auszugeben die Expansion neu angelegter Variablen zu untersuchen.

\subsection*{Aufbau des Makefile}

Das Makefile ist grob in drei Abschnitte unterteilt, wobei der erste
Abschnitt Variablen und Konstanten beinhaltet. Der zweite Abschnitt
enthält die Regeln zur Bestimmung der Abhängigkeiten und allgemein
zur Steuerung des Buildprozesses. Im letzten Abschnitt befinden sich
die PHONY Targets, also die ``Funktionen'' des Makefile, welche
explizit vom Nutzer ausgeführt werden (clean, size, ...).

Allgemein werden im ersten Abschnitt Einstellungen wie das verwendete
Chip-Modell und dessen Taktfrequenz , der verwendete Compiler und
Linker, sowie die Optionen, welche ihnen übergeben werden, vorgenommen.
Darauf folgen grundlegende Einstellungen zum Aufbau des Projektverzeichnisses
gefolgt von Variablen zur Erfassung der Daten im Projektverzeichnis.

Die Regeln zur Bestimmung der Abhängigkeiten basieren auf den für
Make typischen ``automatischen Variablen'', wobei beispielsweise
für jede C-Quelldatei eine Objektdatei angelegt wird. Durch Benutzung
der automatischen Variablen wird auch erreicht, dass Make die Quelldateien
in der richtigen Reihenfolge übersetzt, sodass sichergesetellt ist,
dass bei der Übersetzung einer jeden C-Quelldatei die dafür notwendigen
Objektdateien bereits vorliegen. Um die Übersichtlichkeit zu wahren,
werden die Dateien, die nur temporär zum Build benötigt werden, in
das Unterverzeichnis ``build'' gelegt. Des Weiteren werden die Ausgabedateien
in das Unterverzeichnis ``bin'' gelegt.

Wenngleich die Erstellung der Dokumentation nicht innerhalb des Makefile
realisiert wurde, obwohl auch dies kein Problem darstellt, enthält
das Makefile noch einige Variablen zur Erfassung der Build-Artefakte
der Latex Dokumentation. Mithilfe dieser Variablen kann der make clean
Befehl auch das Unterverzeichnis ``doc'' von temporären Dateien
befreien und so das gesamte Projektverzeichnis sauber halten.

 

